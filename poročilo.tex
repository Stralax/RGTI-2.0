\documentclass[a4paper]{article}

\usepackage[utf8]{inputenc}
\usepackage{erk}
\usepackage{times}
\usepackage{graphicx}
\usepackage[top=22.5mm, bottom=22.5mm, left=22.5mm, right=22.5mm]{geometry}

\usepackage[slovene,english]{babel}
\usepackage{hyperref}
\usepackage{url}

\let\oldfootnotesize\footnotesize
\renewcommand*{\footnotesize}{\oldfootnotesize\scriptsize}

\begin{document}
\title{Poročilo za prvi seminar pri predmetu Računalniška grafika in tehnologija iger}

\author{Strahinja Đorđević$^{1}$, Natalija Ivanović$^{2}$, Nikola Kokotović$^{3}$ }

\affiliation{	$^{1}$Univerza v Ljubljani, Fakulteta za računalništvo in informatiko \\ 
				$^{2}$Univerza v Ljubljani, Fakulteta za računalništvo in informatiko  \\
                $^{3}$Univerza v Ljubljani, Fakulteta za računalništvo in informatiko }

\email{E-pošta: nk37190@student.uni-lj.si}

\maketitle

\selectlanguage{slovene}

\begin{abstract}{Abstract}
Igra predstavlja simulacijo ljubljanske restavracije Ajda. 
Igralec prevzame vlogo kuharja, ki mora sprejemati naročila, jih dostaviti strankam in pobirati denar.
Cilj igre je zadovoljiti čim več strank v čim krajšem času, pri čemer mora igralec paziti na pravilno pripravo jedi in pravočasno dostavo.  
Igra je primerna za vse, ki uživajo v simulacijskih igrah in želijo preizkusiti svoje sposobnosti v vlogi kuharja.
\end{abstract}

\begin{comment}
% -- Zgolj navodila - to v končni verziji dokumenta odstranite.
\section*{Navodila}
Ta dokument naj vam služi kot osnova za pisanje poročila o seminarju pri predmetu. Končno poročilo ne sme vsebovati več kot 4 strani besedila (skupaj s slikami lahko več), ne sme pa biti krajše od dveh strani vključno s slikami. Slike vključite v dokument kot kaže primer s sliko \ref{fig:slika} in se nanje tudi sklicujte. Prav tako v besedilu predstavite vsebino slik.

\begin{figure}[!htb]
    \begin{center}
        \includegraphics[width=\columnwidth]{wolfenstein.jpg}
        \caption{Kratek opis slike.} \label{fig:slika}
    \end{center}
\end{figure}

Pri pisanju poročila vključujte tudi reference na vire s katerimi ste si pomagali pri izdelavi seminarja. To so lahko pisni viri v obliki knjig \cite{Foley1994}, člankov \cite{Meng2015} ali drugih virov, ki jih dodajte med reference, spletne vire pa navajajte v nogi\footnote{\url{https://en.wikipedia.org/wiki/Computer_graphics}}.

Pri izdelavi igre se omejite na izdelavo enega samega nivoja igre, ki pa ga dodelajte in izpilite kolikor vam do\-pu\-šča predviden čas.

% do sem gre ven
\end{comment}



\section{Pregled igre}
Igralec prevzame vlogo kuharja, ki mora sprejemati naročila, jih dostaviti strankam in pobirati denar.
Igra je povprečno zahtevna, za ljudje od 7 do 107 let.
Igralec na začetku od natakarja prevzame naročilo, nato gre v kuhinjo, kjer pripravi naročeno jed. Ko je jed pripravljena, jo dostavi stranki in pobere denar.
Kuhar se mora zelo dobro organizirati kako bi čim več strank zadovoljil v čim krajšem času.
Vsaka narejena jed prinese določeno število točk, ki se seštevajo na koncu igre.

\subsection{Opis sveta}
Svet igre je zasnovan kot kuhinja v restavraciji Ajda, kjer se odvija večina dogajanja. Model kuhinje je realističen, vendar ni povsem enak kuhinji omenjene restavracije. 
V svetu se nahajajo štiri sobe: omenjena kuhinja, shramba, soba za smeti in restavracija. Svet je zaprt, tako da igralec ne more zapustiti stavbe.
Osebki se premikajo v treh dimenzijah.

\subsubsection{Pregled}
Igralec se nahaja v kuhinji in interaktira z natakarjem, in predmeti v kuhinji. Od natakrja prevzame naročilo, nato gre v kuhinjo, kjer pripravi naročeno jed. 
Za pripravo jedi uporavlja razne naprave, ki se nahajajo v kuhinji. Predmeti v kuhinji so štedilnik, friteza, zamrzovalnik, hladilnik in smeti. 
Kuhar tudi mora vzeti potrebne sestavine iz hladilnika in shrambe za pripravo jedi. Na voljo so mu: meso, zelenjava, pomfrit, hleb in Coca-Cola.

\subsubsection{Ozadje}
Ozadje igre predstavlja prostor za stranke restavracije Ajda, kjer se nahajajo mize in stoli.

\subsubsection{Ključne lokacije}
Ključne lokacije so natakarjev pult (kjer kuhar prevzame naročilo), shramba (kjer kuhar lahko najde kruh), zamrzovalnik (kjer kuhar lahko najde meso), hladilnik (kjer kuhar lahko najde Coca-Colo in zelenjavo),
štedilnik (kjer kuhar lahko speče meso), friteza (kjer kuhar lahko speče pomfrit), delovna miza (kjer kuhar sestavi burger), in smeti (kamor kuhar odvrže odpadke). 

\subsubsection{Velikost}
Svet vsebuje 4 sobe - prostor za stranke, zamrzovalnik/shramba, kuhinja in smetišče. 

\subsubsection{Objekti} - NATALIJA
V igri smo vključili različne ključne elemente, kot so kuhinjski aparati, hladilniki, mize, stoli, hrana, pijače in drugi predmeti. 
Večina teh elementov je bila prilagojena našim potrebam, vsak objekt pa je bil posebej uvožen in prilagojen. 
Dodatne teksture smo skrbno izbrali s spletnih mest, kot je freepik.com, medtem ko so bili objekti pridobljeni z naslednjih virov: cgtrader.com, turbosquid.com in sketchfab.com.

\subsubsection{Čas}
Čas igre teče v realnem času.

\subsection{Igralni pogon in uporabljene tehnologije} - STRAHINJA
V poglavju podrobno predstavite katere tehnologije ste uporabili pri izdelavi vašega seminarja. V kolikor ste uporabili kakšno dodatno ogrodje oz. orodje ga na tem mestu predstavite in pojasnite čemu.

\subsection{Pogled} - STRAHINJA
Definirajte kakšen bo pogled v vašo igro. Kakšno kamero boste uporabili, kaj vse bo uporabnik videl, kako boste poudarjali posamezne stvari ipd.

\section{Osebek}
Igralec prevzame vlogo samo kuharja, lahko ga premika po kuhinji, pobira predmete in jih uporablja. Lahko še od natakarja prevzame naročilo.

\section{Uporabniški vmesnik}
Uporabniški vmesnik predstavljajo invetar, ki prikazuje katere sestavine ima kuhar na voljo, trenutna naročila, ki jih mora kuhar pripraviti, število točk, ki jih je kuhar že zbral in čas pečenja mesa.

\section{Glasba in zvok}
Zvoki v igri so prevzeti s spleta. Zvok obstaja za interakcijo z objekti, štedilnik, ko se meso peče, hod kuharja in ambientalni zvoki restavracije.
Glasba v igri je lastno izdelana.

\section{Gameplay}
Igra se začne z naročilom, ki ga kuhar prevzame od natakarja. Nato gre v kuhinjo, kjer pripravi naročeno jed. 
Ko je jed pripravljena, jo dostavi nazaj natakarju. Nato lahko prevzame novo naročilo. 

\section{Zaključki in možne nadgradnje} - NATALIJA
(NIKOLA/STRAHINJA moraćete da dodate ovaj delić) 
Pri izdelavi igre smo se naučili tudi 3D modeliranja, kar nam je omogočilo boljše razumevanje oblikovanja in prilagajanja objektov v prostoru.
Razlike, do katerih je prišlo med začetnim načrtom in končno izvedbo, so se nanašale predvsem na vsebino menija, ki smo ga sprva želeli narediti bolj obsežnega, ter na dodajanje še nekaterih objektov v končni model.
Končna izvedba igre je precej blizu prvotni zamisli, čeprav so tehnologije imele svoje omejitve.

\small
\bibliographystyle{plain}
\bibliography{references}

\end{document}
